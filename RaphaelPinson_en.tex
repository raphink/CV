%% Based off `template.tex'.
%% Copyright 2006-2010 Xavier Danaux (xdanaux@gmail.com).
%%
%% Copyright 2010-2017 Raphaël Pinson (raphink@gmail.com).
%
% This work may be distributed and/or modified under the
% conditions of the LaTeX Project Public License version 1.3c,
% available at http://www.latex-project.org/lppl/.

% Version: 20110122-4


\documentclass[11pt,a4paper,nolmodern]{moderncv}

\usepackage{RaphaelPinson}
\address{Chemin du Jura 3}{1041~Poliez-le-Grand}{Switzerland}

\usepackage[english]{babel}
\linespread{0.9}
% for some reason, lines take up a lot of space in itemize in English...
\newenvironment{tightitemize}
   {\begin{itemize}
   \setlength{\parskip}{0pt}}
   {\end{itemize}}


% personal data
\title{Cloud Native Tech Evangelist}
\extrainfo{%
\linkedin~\httplink{www.linkedin.com/in/raphink}\\%
\octocat~\httplink{www.github.com/raphink}\\%
Driving License} % optional, remove the line if not wanted

\myquote{Freely you have received, freely give}{Matthew 10:8}


%\nopagenumbers{}                             % uncomment to suppress automatic page numbering for CVs longer than one page
%----------------------------------------------------------------------------------
%            content
%----------------------------------------------------------------------------------
\begin{document}
\setmainfont{Minion Pro}
\setsansfont{Myriad Pro}

\hyphenpenalty=10000
\maketitle


\section{Experience}

\tlcventry{2012}{0}
          {Cloud Native Tech Evangelist}
          {\href{http://www.camptocamp.com}{Camptocamp}}
          {Chambéry then Lausanne}
          {}
          {
  \begin{itemize}
    \item \emph{Puppet \& Cloud}\newline
      When I joined Camptocamp, the instructure team's activity focused on
      managing 500+ servers on behalf of clients, using mainly \textbf{Puppet}, \textbf{OpenVZ}, and \textbf{Amazon Web Services}.
    \item \emph{Partnerships \& Training}\newline
      In order to leverage our team's renowned expertise in Puppet,
      I set up a \textbf{strategic partnership} with Puppet Labs and started teaching the official
      Puppet curriculum. This required me to be certified by both the \textbf{Puppet Professional} and \textbf{Puppet Developer} certifications.
      I also wrote our own curriculum based on \textbf{Foreman} for clients using Open Source Puppet, 
      as well as an \textbf{Augeas} curriculum.\newline
      Since then, I have been a \textbf{trainer} to many engineers at major banks, universities (EPFL, Unibe), research facilities (CERN, CNRS), and many more.
    \item \emph{Open Source Work}\newline
      As our team managed tens of public open-sourced Puppet modules on \textbf{GitHub}, I strived to improve their quality
      by implementing industry standards, \textbf{unit and acceptance tests}, publishing them on the Puppet forge, and getting them approved by Puppet Labs as standard implementations. I also managed the \textbf{Augeasproviders} project and maintained the Puppet modules associated with it.
      My Puppet expertise led to many \textbf{consulting} missions, ranging from architecture to migration,
      as well as specific \textbf{Ruby} development, and earned me a recognition as \textbf{Puppet Champion of the year 2020}, with a title of \textbf{Extraordinary Puppeteer}.
    \item \emph{Terraform}\newline
      With our use of public clouds broadening, we switched our provisioning from \textbf{Cloud Formation} to \textbf{Terraform} and contributed multiple providers and modules to the community.
      I also authored \textbf{Terraboard}, a web UI to visualize Terraform state information, programmed in \textbf{Go} and \textbf{AngularJS}.
    \item \emph{Containers \& Orchestration}\newline
      Around 2014, we moved our Puppet acceptance tests from \textbf{Vagrant} to \textbf{Docker}.
      We quickly realized Docker's potential, and decided to implement our Puppet 4 stack on top of it, in order to simplify its architecture.
      Starting with Docker-Compose, we switched to \textbf{Rancher} in 2015 to scale the infrastructure horizontally, and became the \textbf{first Rancher partner in Europe}.
      Around that time, we wrote multiple tools to improve containers management, in particular \textbf{Bivac}, a tool to automate container volumes backup.
      Finally, we migrated the Puppet stack to \textbf{OpenShift} around 2018. \newline
      \textbf{Kubernetes} expertise has since become our main activity. I have co-authored a 5-day containers curriculum covering both Docker and Kubernetes and taught it multiple times.
      In the meanwhile, I have contributed to getting Camptocamp set up as one of the first \textbf{Kubernetes Certified Service Provider} (KCSP) in Switzerland and the first \textbf{Kubernetes Training Partner} (KTP) in Europe.\newline
      In that context, I have completed the \textbf{OpenShift Administration}, \textbf{CKAD}, \textbf{CKA}, and \textbf{CKS} certification programs.
    \item \emph{Sales \& Marketing}\newline
      As an expert in Configuration Management, Cloud, and Containers Orchestration, I have often 
      supported our \textbf{sales} force and contributed to multiple \textbf{tenders}.
      My attachment to Camptocamp's image and values has also encouraged me to improve \textbf{marketing} (through market studies, newsletters, blog posts, videos, white papers, designing illustrations, totems \& goodies).
      I have represented Camptocamp at many \textbf{conferences and meetups} (both physical and online) on various subjects (Puppet and its ecosystem, Cloud, DevOps, etc.).
  \end{itemize}
}

\tlcventry{2006}{2012}{Systems Engineer}{\href{http://www.orness.com}{ORNESS} then \href{http://www.alten.fr}{Alten}}{Sophia Antipolis}{}{Consultant at France T\'el\'ecom
\begin{itemize}
  \item \emph{Cfengine}\newline
    I was originally hired to migrate about 3000 machines from \textbf{Cfengine} 1 to Cfengine 2.
    This was an opportunity for me to learn about \textbf{Configuration Management} principles and to discover \textbf{Puppet} along the way.
  \item \emph{Package build farm \& repository}\newline
    Migration to Cfengine 2 was an opportunity to standardize software packages used.
    I set up a new automated Debian package repository using \textbf{reprepro}
    and a \textbf{Debian buildd} farm to compile the packages for multiple distributions and architectures. Both i386 and amd64 builders ran on a single machine, using \textbf{LXC} containers as light virtualization to isolate contexts.\newline
    I encouraged developers to package their applications and helped them set up \textbf{CI/CD} pipelines using \textbf{Hudson} (now \textbf{Jenkins}) to automatically release their packages to the repository, using \textbf{GnuPG} keys as the authentication layer.\newline
    In order to encourage good practices, I taught multiple sessions of a \textbf{Debian packaging curriculum to colleagues.}
  \item \emph{Gforge}\newline
    I administered a Gforge platform for hundreds of developers.
    As part of this task, I set up a replicated fallback server and wrote a CVS replication software using \textbf{Perl} and \textbf{inotify}.
  \item \emph{Policy Conformity}\newline
    In order to improve comformity to standards as we adopted \textbf{ITIL} processes,
    I designed a \textbf{Treetester}, a Perl software working as a sieve of tests to identify priority actions to take on machines and platforms and report to platform administrators in a Web UI.
\end{itemize}
}


\tlcventry{2008}{0}{Developer}{\href{http://www.augeas.net}{Augeas}}{Internet}{}{Development, bugfix and documentation
\begin{tightitemize}%
 \item Writing of Augeas lenses;
 \item Coding in C;
 \item Improvement of autotools configuration;
 \item NaturalDocs Integration (documentation generator);
 \item International Conference Speaker (Belgium (\href{http://archive.fosdem.org/2009/schedule/events/fedora_augeas}{FOSDEM})).
\end{tightitemize}}

\tlcventry{2005}{0}{\href{https://launchpad.net/~raphink}{Developer/Maintainer}}{\href{http://www.ubuntu.com}{Ubuntu}}{Internet}{}{Development, bugfix and documentation
\begin{tightitemize}%
 \item Creation, maintenance and review of software packages;
 \item Writing of technical documentation for developers;
 \item Maintenance of the QA system for software package;
 \item Bug management;
 \item International Conference Speaker (Hungary, Germany (\href{http://www.linuxtag.org/}{LinuxTag})).
\end{tightitemize}}


\tlcventry{2005}{0}{Open-Source Contributor}{Various projects}{Internet}{}{%
\begin{tightitemize}%
 %\item \href{https://launchpad.net/wavebiblebot}{Flammard Bible Bot}: Google Wave Bot, a personal project written in Python/AppEngine;
 %\item \href{http://www.ichthux.com}{Ichthux}: Creation and maintenance of a specialized Linux distribution based on Ubuntu;
 %\item \href{https://launchpad.net/byobu}{Byobu}: contribution of scripts and patches in Python and Bash;
 \item \href{http://search.cpan.org/~raphink}{CPAN Author}: maintainer of several Perl modules on CPAN;
 \item \href{http://www.ctan.org/author/id/pinson}{CTAN Author}: maintainer of several \LaTeX{} packages on CTAN.
\end{tightitemize}}


\subsection{Other Experience}

\tlcventry{2009}{0}{Translator and Editor}{La Colombe Calvary}{Nice}{}{%
\begin{tightitemize}%
 \item Translation of English theology books into French;
 \item Edition of books using \LaTeX{}, \XeTeX{} and \LuaTeX{};
 \item Publishing of books using on-demand publishing services (CreateSpace and Lulu).
\end{tightitemize}}

% Restore normal labels
%\tltext{\scriptsize}

%\tldatelabelcventry{2004}{July 2004}{Blue Collar Internship}{\href{http://www.snecma.com}{SNECMA}}{Moissy-Cramayel}{}{Assembled and equilibrated turbo reactors for planes}

%\tldatelabelcventry{2002}{Summer 2002}{Fire Safety Officer}{\href{http://www.euroguard.fr/}{Euroguard}}{Marcoussis}{}{Supervised an Alcatel research site}

%\tldatelabelcventry{2001}{Summer 2001}{Surveillance Agent}{Penauille Polys\'ecurit\'e}{Paris}{}{Supervised the headquarters of the French Red Cross}

%\tllabelcventry{1999}{2000}{1999--2000}{Certified First Responder}{\href{http://www.croix-rouge.fr/}{French Red Cross}}{Paris Suburbs}{}{Served in several volunteer missions as a paramedic}


\pagebreak

\section{Education}

\tldatecventry{2013}{Puppet Trainer}{\href{https://puppetlabs.com/services/puppet-training/}{PuppetLabs}}{Amsterdam}{}{Official Training Partner for PuppetLabs' courses (Puppet Fundamentals, Puppet Advanced, Extending Puppet with Ruby)}

\tldatecventry{2009}{ITIL{\LARGE\textregistered} v3 Foundation}{\href{http://www.itil-officialsite.com/home/home.asp}{EXIN}}{Sophia Antipolis}{}{Organization and efficiency of the Information System}

\tldatecventry{2005}{Student in Business Creation and Management}{\href{http://www.creation-transmission.com/}{Cr\'eafort}}{Poitiers}{}{Accounting, Management, Marketing, Law}

\tldatecventry{2005}{Student in Pedagogy (`Gestion Mentale')}{\href{http://www.ifgm.org/}{IFGM}}{Bordeaux}{}{Pedagogy, Didacticism, Cognitive Psychology}

\tllabelcventry{2003}{2005}{2003--2005}{Student in Mechanical Engineering}{\newline\href{http://www.ensma.fr}{\'Ecole Nationale Sup\'erieure de M\'ecanique et d'A\'erotechnique (ENSMA)}}{Poitiers}{}{Solid \& Fluid Mechanics, Mathematics, Combusion \& Propulsion, Materials Science, Automation, Signal Processing, Computer Sciences, Management}

\tldatecventry{2004}{Licence, equivalent of a British Bsc, in Mechanics}{\href{http://www.univ-poitiers.fr/}{Universit\'e de Poitiers}}{Poitiers}{}{Solid \& Fluid Mechanics, Mathematics, Combusion \& Propulsion, Materials Science, Automation, Signal Processing, Computer Sciences, Management}

\tllabelcventry{2002}{2003}{2002--2003}{Student in Chemistry}{\href{http://www.u-psud.fr}{Universit\'e de Paris XI}}{Orsay}{}{Chemistry, Physics, Mathematics}




\section{Certifications}

\tldatecventry{2021}{Certified Kubernetes Security Specialist (CKS)}{\href{https://training.linuxfoundation.org/certification/verify/}{CNCF}}{LF-xpar0qezpj}{}{Kubernetes}

\tldatecventry{2020}{Certified Kubernetes Application Developer (CKAD)}{\href{https://training.linuxfoundation.org/certification/verify/}{CNCF}}{LF-gikljn6k0e}{}{Kubernetes}

\tldatecventry{2019}{Red Hat Certified Specialist in OpenShift Administration}{\href{https://www.redhat.com/en/services/training/ex280-red-hat-certified-specialist-in-openshift-administration-exam}{Red Hat}}{190-035-931}{}{OpenShift}

\tldatecventry{2018}{Certified Kubernetes Administrator (CKA)}{\href{https://training.linuxfoundation.org/certification/verify/}{CNCF}}{LF-vqymlq6gyo}{}{Kubernetes}

\tldatecventry{2013}{Puppet Certified Developer}{\href{https://puppetlabs.com/services/certification/puppet-developer/}{PuppetLabs}}{PCD0000011}{}{Puppet IT automation software}

\tldatecventry{2013}{Puppet Certified Professional}{\href{https://puppetlabs.com/services/certification/puppet-professional/}{PuppetLabs}}{PCP0000116}{}{Puppet IT automation software}

\tldatecventry{2009}{ITIL{\LARGE\textregistered} v3 Foundation Examination}{\href{http://www.itil-officialsite.com/home/home.asp}{EXIN}}{00055512}{}{Organization and efficiency of the Information System}


\section{Foreign Languages}
\cvlanguage{French}{Native}{Mother Tongue}
\cvlanguage{English}{Fluent}{Daily practice, conferences given in English}
\cvlanguage{Spanish}{Good Level}{Occasional practice}
\cvlanguage{German}{Good Level}{Studied 9 years in school}
%\cvlanguage{Russian}{Intermediary Level}{Studied 3 years in school}
%\cvlanguage{Dutch}{Beginner}{Studied alone}
%\cvlanguage{Swedish}{Beginner}{Studied alone}

\section{Personal interests}

\cvhobby{Music}{Clarinet, guitar, piano, organ, saxophone}
\cvhobby{Sports}{Hiking, sailing, golf, climbing}
\cvhobby{Others}{Genealogy, reading}


%\renewcommand{\listitemsymbol}{-} % change the symbol for lists


% Publications from a BibTeX file without multibib\renewcommand*{\bibliographyitemlabel}{\@biblabel{\arabic{enumiv}}}% for BibTeX numerical labels
%\nocite{*}
%\bibliographystyle{plain}
%\bibliography{publications}       % 'publications' is the name of a BibTeX file

% Publications from a BibTeX file using the multibib package
%\section{Publications}
%\nocitebook{book1,book2}
%\bibliographystylebook{plain}
%\bibliographybook{publications}   % 'publications' is the name of a BibTeX file
%\nocitemisc{misc1,misc2,misc3}
%\bibliographystylemisc{plain}
%\bibliographymisc{publications}   % 'publications' is the name of a BibTeX file

\end{document}

